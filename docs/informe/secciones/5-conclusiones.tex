\section{Conclusiones}

Todo método tiene sus ventajas y sus desventajas. La primer gran conclusión que podemos obtener de este trabajo es que utilizar PageRank para ligas deportivas puede ser muy eficiente. A diferencia del método utilizado por la AFA, toma en cuenta las posiciones en las que se encuentran los equipos a enfrentarse. Esto genera que si un equipo que se encuentra en una posición del ranking muy baja gana un partido con uno que esta entre los primeros entonces el vencedor subirá mas en la tabla que lo que podría subir si le dan solamente 3 puntos. Por otro lado en PageRank, importa cual es la diferencia de goles, en cambio, con el método de la AFA es indistinta.

Comparando las posiciones entre dos fechas, utilizando el método de PageRank, puede pasar que se genere un salto con algún equipo. Es decir, un equipo puede pasar de estar en las últimas posiciones de la tabla a una de las primeras.

El método utilizado por la AFA es mas sencillo, porque en PageRank es muy difícil especular con los resultados ya que toma en cuenta no solo si ganan o pierden los equipos si no que se basa en mas información para determinar el ranking. Como las ligas deportivas son vistas por muchas personas, puede generar problemas que no todos comprendan el método para calcular posiciones en la tabla. 