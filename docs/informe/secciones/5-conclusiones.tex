\section{Conclusiones}

	Todo método tiene sus ventajas y sus desventajas. La primer gran conclusión que podemos obtener de este trabajo es que utilizar \emph{PageRank} para ligas deportivas puede ser muy efectivo. A diferencia del método utilizado por la \acr{AFA}, toma en cuenta las posiciones en las que se encuentran los equipos a enfrentarse. Esto genera que si un equipo que se encuentra en una posición del \emph{ranking} muy baja gana un partido con uno que esta entre los primeros entonces, el vencedor subirá más en la tabla que lo que podría subir si le dan solamente 3 puntos. Por otro lado en \emph{PageRank}, importa cual es la diferencia de goles, en cambio, con el método de la \acr{AFA} es indistinta.

	Comparando las posiciones entre dos fechas, utilizando el método de \emph{PageRank}, puede pasar que se genere un salto con algún equipo. Es decir, un equipo puede pasar de estar en las últimas posiciones de la tabla a una de las primeras.

	El método utilizado por la \acr{AFA} es mas sencillo, porque en \emph{PageRank} es muy difícil especular con los resultados, ya que toma en cuenta no solo si ganan o pierden los equipos si no que se basa en más información para determinar el \emph{ranking}. Como las ligas deportivas son vistas por muchas personas, puede generar problemas que no todos comprendan el método para calcular posiciones en la tabla. 
