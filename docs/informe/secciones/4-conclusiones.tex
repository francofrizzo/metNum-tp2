\section{Conclusiones}

	Durante la realización de este trabajo, se pudo observar un ejemplo más de cómo las herramientas teóricas que brinda el álgebra lineal pueden tener aplicaciones sumamente prácticas. El éxito del método de \emph{PageRank} es indiscutible, ya que su gran precisión a la hora de seleccionar resultados de relevancia fue crucial para posicionar al buscador Google por sobre sus competidores y convertirlo en el más utilizado a nivel mundial.

	En cuanto a su adaptación para la clasificación en ligas deportivas, \emph{GeM}, se encontraron múltiples razones por las que puede resultar muy efectivo. A diferencia del método utilizado por la \acr{AFA}, toma en cuenta las posiciones en las que se encuentran los equipos a enfrentarse. Esto genera que si un equipo que se encuentra en una posición del \emph{ranking} muy baja gana un partido con uno que esta entre los primeros, el vencedor subirá más en la tabla que lo que podría subir si le dan solamente una cantidad fija de puntos. Por otro lado en \emph{GeM}, importa cual es la diferencia de goles, en cambio, con el método de la \acr{AFA} es indistinta.

	Una de las características que se observa al utilizar el método \emph{GeM} es que, comparando las posiciones entre dos fechas consecutivas, puede pasar que se genere un salto con algún equipo. Es decir, un equipo puede pasar de estar en las últimas posiciones de la tabla a una de las primeras, lo cual podría generar sorpresas, en ocasiones desagradables.

	Utilizando la información que nos brindan los experimentos realizados, podríamos corregir el método \emph{GeM} tomando un buen criterio ante la situación de empate. Por lo que pudimos observar a lo largo del trabajo una buena opción es darle un determinado puntaje a cada uno de los equipos empatados evitando que éste sea demasiado alto y pueda suceder que empatar sea mas conveniente que ganar. Por otro lado, si fuera cero entonces hay determinados casos, como cuando un equipo fuerte empata contra uno débil, en donde no dar puntos puede ser muy injusto para el equipo endeble. En este caso lo que no podemos evitar es que al empatar un equipo débil con uno fuerte el débil ascienda demasiadas posiciones en el ranking de la liga. Por ello se decidió experimentar que sucede si tomamos un parámetro que determine la importancia que se le dará a la situación de empate. En este trabajo notamos que de esta forma empatar y perder genera la misma cantidad de puntos. Dejamos a trabajo futuro determinar el por que de esta situación. 
	
	El método de \emph{GeM} tiene la fortaleza de ser robusto ante intentos de especular con los resultados, ya que toma en cuenta no solo si ganan o pierden los equipos si no que se basa en más información para determinar el \emph{ranking}. No obstante, su complejidad también es una de las desventajas que puede atribuirsele: claramente, el método utilizado por la \acr{AFA} es mas sencillo, y, como las ligas deportivas son vistas por muchas personas, puede generar problemas que el método utilizado para calcular posiciones en la tabla sea extremadamente difícil de comprender. 

