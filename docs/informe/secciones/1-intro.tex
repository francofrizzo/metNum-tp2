\section{Introducción}

    En la actualidad, Internet ocupa un lugar central en la vida cotidiana de millones de personas, y una de las principales tareas que estas llevan a cabo en línea es la búsqueda de información. A eso se debe el rol central que ocupan los buscadores, y la gran importancia de contar con algoritmos que permitan indexar y clasificar la información disponible en la red y, muy especialmente, seleccionar de este enorme volumen de datos solo aquellos que sean más relevantes para un determinado contexto de búsqueda.

    Por otro lado, las partidos de las grandes ligas deportivas movilizan a enormes masas de seguidores a lo largo y a lo ancho de todo el mundo, y además, importantes intereses económicos dependen de sus resultados. Por lo tanto, es de gran interés que estos resultados puedan definirse con criterios justos y objetivos, que no sea preseten a la especulación y sean robustos ante posibles intentos de manipulación.

    Si bien a primera vista estos dos problemas pueden parecer poco relacionados, poseen características en común: en ambos casos, se busca dar un orden de importancia a una determinada colección de elementos, organizándolos en un \emph{ranking} en base información estadística que puede obtenerse sobre las relaciones existentes entre ellos.

    En este trabajo, se introducirá uno de los métodos que ha sido utilizado con gran éxito para resolver el primero de los problemas mencionados: el algoritmo de \emph{PageRank}, presentado en 1998 por Sergey Brin y Larry Page\cite{Brin1998}, que constituye una parte esencial del motor del reconocido buscador Google. Tras una introducción teórica a los conceptos matemáticos que fundamentan este método, se presentará una implementación del mismo y se expondrán resultados experimentales con el fin de analizar su eficiencia y su capacidad para resolver el problema planteado.

    Por otra parte, y de forma paralela, se procederá a abordar la situación planteada por el problema de la clasificación de equipos en ligas deportivas. Se mostrará que, tal y como se expone en Govan et al.\cite{Govan2008}, \emph{PageRank} puede ser adaptado sin mayores dificultades para emplearse como criterio de clasificación en este contexto, y se analizará el caso particular de su utilización para elaborar la tabla de posiciones del torneo de Primera Divisón de la Asociación del Fútbol Argentino (AFA), realizando experimentos para evaluar su desempeño en comparación con el sistema de puntajes actual, y la forma en la que su funcionamiento ve afectado por variaciones en los parámetros del algoritmo.

        \subsection{Conceptos teóricos}

        En esta sección, enunciamos a modo introductorio los principales conceptos que funcionan como sustento teórico de los métodos numéricos que emplearemos. Nuestro objetivo no es profundizar en los mismos, dado que ya existen excelentes trabajos realizados que abordan cada uno de ellos más en detalle.

        Consideremos un sistema que puede tomar diferentes estados, dentro de un conjunto finito de ellos. Un \emph{proceso estocástico} es una sucesión de estos estados. Si consideramos un caso particular de proceso estocástico, en que la probabilidad de que el sistema se encuentre en un estado dado en el momento $k+1$ queda completamente determinada por su estado en el momento $k$, tenemos una \emph{cadena de Márkov}.

        Supongamos que el sistema tiene $n$ estados posibles, y denominemos $p_{ij}$ a la probabilidad de que el sistema pase del estado $j$ al estado $i$. Entonces podemos acomodar estos valores en una matriz $P \in \mathbb{R}^{n \times n}$, que llamaremos \emph{matriz de transición} de la cadena.

        Notemos que todas las entradas de la matriz, dado que indican valores de una probabilidad, son no negativas, y además, para $j = 1, \cdots, n$, tenemos que $\sum_{i=1}^{n} p_{ij} = 1$. A una matriz que cumpla con estas condiciones la denominaremos \emph{matriz estocástica por columnas}, y, como enuncian Bryan y Leise en \cite{Bryan2006}, cumplen con la importante propiedad de tener a $1$ como autovalor dominante; es decir, $1$ es autovalor y además, para cualquier otro $\lambda \in \mathbb{R}$ autovalor de una de estas matrices, se cumple que $|\lambda| < 1$. Si, además, una matriz estocástica cumple que todas sus entradas son estrictamente positivas, entonces podemos afirmar que el espacio de autovectores asociados al autovalor $1$ tiene dimensión $1$.

        En este trabajo, será de interés para nosotros poder calcular este autovector. Se cuentan con métodos muy diversos para realizarlo, pero muchos de ellos son excesivamente costosos, especialmente cuando las matrices son de dimensiones considerables. Afortunadamente, para matrices con las características que acabamos de mencionar, el autovector principal puede calcularse mediante un método iterativo, muy sencillo de formular, conocido como \emph{método de la potencia}. Este método se basa en el siguiente resultado: si $x$ es el autovector principal de la matriz $A$, y $v$ es un vector inicial cualquiera, entonces $\lim_{k\to\infty}A^k v = x$. Es decir, partiendo de un vector inicial y multiplicando repetidas veces por la matriz $A$, el resultado eventualmente convergerá al autovector buscado. Para una demostración de este hecho, como así también una exposición más detallada del proceso, puede consultarse \cite{Kamvar}[Section 3]. Como podrá verse en la sección siguiente, y comprobarse luego experimentalmente, este algoritmo resulta muy apropiado para resolver los problemas planteados en este trabajo, y es capaz de aprovechar características propias de las matrices con las que trataremos para mejorar su eficiencia.
