\section{Introducción}

    En la actualidad, Internet ocupa un lugar central en la vida cotidiana de millones de personas, y una de las principales tareas que estas llevan a cabo en línea es la búsqueda de información. A eso se debe el rol central que ocupan los buscadores, y la gran importancia de contar con algoritmos que permitan indexar y clasificar la información disponible en la red y, muy especialmente, seleccionar de este enorme volumen de datos solo aquellos que sean más relevantes para un determinado contexto de búsqueda.

    Por otro lado, las partidos de las grandes ligas deportivas movilizan a enormes masas de seguidores a lo largo y a lo ancho de todo el mundo, y además, importantes intereses económicos dependen de sus resultados. Por lo tanto, es de gran interés que estos resultados puedan definirse con criterios justos y objetivos, que no sea preseten a la especulación y sean robustos ante posibles intentos de manipulación.

    Si bien a primera vista estos dos problemas pueden parecer poco relacionados, poseen características en común: en ambos casos, se busca dar un orden de importancia a una determinada colección de elementos, organizándolos en un \emph{ranking} en base información estadística que puede obtenerse sobre las relaciones existentes entre ellos.

    En este trabajo, se introducirá uno de los métodos que ha sido utilizado con gran éxito para resolver el primero de los problemas mencionados: el algoritmo de \emph{PageRank}, presentado en 1998 por Sergey Brin y Larry Page\cite{Brin1998}, que constituye una parte esencial del motor del reconocido buscador Google. Tras una introducción teórica a los conceptos matemáticos que fundamentan este método, se presentará una implementación del mismo y se expondrán resultados experimentales con el fin de analizar su eficiencia y su capacidad para resolver el problema planteado.

    Por otra parte, y de forma paralela, se procederá a abordar la situación planteada por el problema de la clasificación de equipos en ligas deportivas. Se mostrará que, tal y como se expone en Govan et al.\cite{Govan2008}, \emph{PageRank} puede ser adaptado sin mayores dificultades para emplearse como criterio de clasificación en este contexto, y se analizará el caso particular de su utilización para elaborar la tabla de posiciones del torneo de Primera Divisón de la Asociación del Fútbol Argentino (\acr{AFA}), realizando experimentos para evaluar su desempeño en comparación con el sistema de puntajes actual, y la forma en la que su funcionamiento ve afectado por variaciones en los parámetros del algoritmo.


        
